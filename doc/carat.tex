%%%%%%%%%%%%%%%%%%%%%%%%%%%%%%%%%%%%%%%%%%%%%%%%%%%%%%%%%%%%%%%%%
\Chapter{Interface to Carat}

This share package provides a {\GAP} interface to Carat, 
a package of programs for the computation with crystallographic 
groups. Carat itself is implemented in C, and has been developed at 
Lehrstuhl B f{\accent127 u}r Mathematik, RWTH Aachen. It runs on many 
Unix systems.

The {GAP} interface consists of two parts, low level interface 
routines to Carat functions on the one hand, and comfortable high 
level {\GAP} functions on the other hand. The high level functions,
implemented in terms of the low level functions, provide actually
methods for functions and operations declared in the {GAP} library.


%%%%%%%%%%%%%%%%%%%%%%%%%%%%%%%%%%%%%%%%%%%%%%%%%%%%%%%%%%%%%%%%%
\Section{Action from the left and from the right}

In crystallography, the convention usually is that matrix groups
act from the left on column vectors. This convention is adopted
also in Carat. The low level interface routines described below
must respect this convention and provide Carat with data in the 
expected format.

On the other hand, in {\GAP} the convention is that all groups 
act from the right, in the case of matrix groups on row vectors. 
However, in order to make {\GAP} accessible to crystallographers,
many functions important in crystallography are provided in two
variants, one for each convention. This is the case for all high
level routines provided by this package. Where necessary, these high
level routines translate to and from the convention used by Carat.

%%%%%%%%%%%%%%%%%%%%%%%%%%%%%%%%%%%%%%%%%%%%%%%%%%%%%%%%%%%%%%%%%
\Section{Carat input and output files}

Carat routines read their input from one or several input files,
and write the result to standard output. In order to use Carat
routines from within {\GAP}, the input must be prepared in suitably
formatted input files. A Carat command is then executed with these
input files, with standard output redirected to an output file, 
which is read back into {\GAP} afterwards. This section describes
routines interfacing with Carat input and output files.

Working with Carat requires many temporary files. When the Carat
package is loaded, a temporary directory is created, where one can
put such files. The routine

\>CaratTmpFile( <filename> ) F

returns a file name <filename> in the Carat temporary directory, which
can be used to store temporary data. Of course, it is also possible
to use any other file name, for instance files in the current directory.

\>CaratShowFile( <filename> ) F

displays the contents of any text file on the terminal. This can be
used to inspect the contents of Carat input and output files.

Most Carat data files are in either of two formats. The first Carat
file type is the Matrix File, containing one or several matrices.
The following routines serve as interface to Carat Matrix Files.

\>CaratWriteMatrixFile( <filename>, <data> ) F

takes a file name and a matrix or a list of matrices, and writes the
matrix or matrices to the file.

\>CaratReadMatrixFile( <filename> ) F

reads a Carat matrix file, and returns a matrix or a list of matrices
read from the file.

The second Carat file type is the Bravais File, containing information
on a finite unimodular group. In { \GAP}, the contents of a Bravais File 
is represented by a Bravais record, having the following components:
\beginitems
`generators'   & generators of the finite unimodular group

`formspace'    & basis of the space of invariant forms (optional)

`centerings'   & list of centering matrices (optional)

`normalizer'   & additional generators of the normalizer in GL(n,Z) (optional)

`centralizer'  & additional generators of the centralizer in GL(n,Z) (optional)

`size'         & size of the unimodular group (optional)
\enditems

The following routines serve as interface to Carat Bravais Files.

\>CaratWriteBravaisFile( <filename>, <data> ) F

takes a file name and a Bravais record, and writes the data in the
Bravais record to the file.

\>CaratReadBravaisFile( <filename> ) F

reads a Bravais File, and returns the resulting Bravais record.

Certain Carat programs produce output files in a special format.
These files usually consist of a varying number of header lines, 
followed by the data of several Bravais records. The header lines 
are currently just discarded.

\>CaratReadQtoZFile( <filename> ) F

reads the output file of Carat program QtoZ, and returns the
resulting list of Bravais records. 

\>CaratReadBravaisInclusionsFile( <filename> ) F

reads the output file of Carat program Bravais\_inclusions, 
and returns the resulting list of Bravais records. 


%%%%%%%%%%%%%%%%%%%%%%%%%%%%%%%%%%%%%%%%%%%%%%%%%%%%%%%%%%%%%%%%%%%
\Section{Executing Carat commands}

To execute a Carat program from within {\GAP}, some low level,
general purpose routines are provided in this share package. 
Higher level routines for certain Carat functions may be available 
in the {\GAP} library or in other share packages. These higher
level functions are expected to use the following low level routines,
so that changes in the low level interface will be transparent. 

An arbitrary Carat program can be executed with the routine

\>CaratCommand( <command>, <args>, <outfile> ) F

where <command> is the name of a Carat program, <args> is a string
containing the command line arguments of the Carat program,
and <outfile> is the name of the file to which the output is to be 
written. Example:

\beginexample
    gap> CaratCommand( "Z_equiv", "file1 file2", "file.out" );
\endexample

A short description of the arguments and options of any Carat 
program can be obtained from the Carat online help facility with

\>CaratHelp( <command> ) F

where <command> is the name of the Carat program. CaratHelp executes
the program with the `-h' option, and writes the output to the 
terminal. Example:

\beginexample
    gap> CaratHelp( "Z_equiv" );
\endexample

%%%%%%%%%%%%%%%%%%%%%%%%%%%%%%%%%%%%%%%%%%%%%%%%%%%%%%%%%%%%%%%%%%%
\Section{Methods provided by Carat}

Carat implements methods for the following functions and operations
declared in the {\GAP} library. For a detailed description of these
functions, please consult the {\GAP} manual (section "Matrix Groups
in Characteristic 0").

\index{IsBravaisGroup}
\index{IsBravaisGroupOnRight}
\index{IsBravaisGroupOnLeft}

\index{BravaisGroup}
\index{BravaisGroupOnRight}
\index{BravaisGroupOnLeft}

\index{BravaisSubgroups}
\index{BravaisSubgroupsOnRight}
\index{BravaisSubgroupsOnLeft}

\index{BravaisSupergroups}
\index{BravaisSupergroupsOnRight}
\index{BravaisSupergroupsOnLeft}

\index{NormalizerInGlnZBravaisGroup}
\index{NormalizerInGlnZBravaisGroupOnRight}
\index{NormalizerInGlnZBravaisGroupOnLeft}

\index{Normalizer!in GLnZ}
\index{Centralizer!in GLnZ}
\index{ZClassRepsQClass}
\index{RepresentativeOperation!in GLnZ}

\begintt
IsBravaisGroup
IsBravaisGroupOnRight
IsBravaisGroupOnLeft

BravaisGroup
BravaisGroupOnRight
BravaisGroupOnLeft

BravaisSubgroups
BravaisSubgroupsOnRight
BravaisSubgroupsOnLeft

BravaisSupergroups
BravaisSupergroupsOnRight
BravaisSupergroupsOnLeft

NormalizerInGLnZBravaisGroup
NormalizerInGLnZBravaisGroupOnRight 
NormalizerGLnZBravaisGroupOnLeft 

Normalizer(  GL(<n>, Integers), <G> ) 
Centralizer( GL(<n>, Integers), <G> )

ZClassRepsQClass

RepresentativeOperation( GL(<n>,Integers), <G1>, <G2> )
\endtt





