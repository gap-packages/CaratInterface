
%%%%%%%%%%%%%%%%%%%%%%%%%%%%%%%%%%%%%%%%%%%%%%%%%%%%%%%%%%%%%%%%%%%%
\Chapter{Methods provided by Carat}

%%%%%%%%%%%%%%%%%%%%%%%%%%%%%%%%%%%%%%%%%%%%%%%%%%%%%%%%%%%%%%%%%%%%
\Section{Rational and Integer Matrix Groups}


The following sections describes properties, attibutes and operations
specific to rational and, in particular, integer matrix groups. They
are usually declared for arbitrary matrix groups, but make sense only
for rational matrix groups, or even only for integer matrix groups.
If they are used with other matrix groups, an error message is
returned.

\> IsRationalMatrixGroup( <G> ) 

checks whether <G> is a rational matrix group.

\> IsIntegerMatrixGroup( <G> )

checks whether <G> is an integer matrix group. Integer matrix
groups are usually called unimodular groups.

\> InvariantLattice( <G> )

determines a basis, with repect to which the rational matrix group
<G> is integer. As every finite rational matrix group is conjugate 
to an integer matrix group, such a basis alway exists if <G> is
finite. If <G> is infinite, such a basis may or may not exist.
In the latter case, `fail' is returned.


%%%%%%%%%%%%%%%%%%%%%%%%%%%%%%%%%%%%%%%%%%%%%%%%%%%%%%%%%%%%%%%%%%%%
\Section{Bravais groups}

If <G> is a unimodular group, it defines a cone of positive definite 
quadratic forms which satisfy $Q = g Q g^{tr}$ for all <g> in <G>. 
The maximal unimodular group leaving this same cone invariant is
called the Bravais group of <G> (and is finite, too).  A finite 
unimodular group <G> is called a Bravais group if it coincides 
with its Bravais group.

\> IsBravaisGroup( <G> )

checks whether <G> is a Bravais group. 

\> BravaisGroup( <G> )

returns the Bravais group of the finite unimodular group <G>.

*Warning:* In the literature, Bravais groups usually are defined with
respect to the cone of positive definite quadratic forms invariant
under $Q \to g^{tr} Q g$ for all <g> in <G>. Since in {\GAP} matrices 
by convention act from the right on row vectors, our definition is
the appropriate one in the {\GAP} context. If the usual Bravais group
of <G> is needed, it can be obtained with

\)TransposedMatrixGroup( BravaisGroup( TransposedMatrixGroup( <G> ) ) ) 

The following two functions determine Bravais subgroups and
supergroups of a Bravais group. They use a catalog of all 
Bravais groups, which currently exists only up to dimension 6.

\> BravaisSubgroups( <G> )

returns the list of all Bravais subgroups of the Bravais group 
<B> of <G>.

\> BravaisSupergroups( <G> )

returns the list of all Bravais supergroups of the Bravais group 
<B> of <G>.


%%%%%%%%%%%%%%%%%%%%%%%%%%%%%%%%%%%%%%%%%%%%%%%%%%%%%%%%%%%%%%%
\Section{Normalizer and Centralizer in GL(n,Z)}

*Warning:* The functions in this section are feasible only if the 
dimension of the group is not too high, not more than 6, say, 
or if the space of invariant quadratic forms has low enough
dimension. It is this latter which actually determines the amount
of work to be done.
\def\cp{)}

%\> Normalizer( GL(<n>,Integers\cp, <G> )!{in GL(n,Z)}
\> `Normalizer( GL(<n>,Integers), <G> )'%
{Normalizer in GL}@{`Normalizer', in GL(n,Z)}

returns the normalizer in <GL(n,$\Z$)> of the finite unimodular group
<G>.

\> NormalizerGLnZBravaisGroup( <G> )

returns the normalizer in <GL(n,$\Z$)> of the Bravais group of the
finite unimodular group <G>, without actually computing the
Bravais group. If <G> knows its Bravais group <B>, the result is 
stored as an attribute of <B>.

\> `Centralizer( GL(<n>,Integers), <G> )'%
{Centralizer in GL}@{`Centralizer', in GL(n,Z)}

returns the centralizer in <GL(n,$\Z$)> of the finite unimodular group
<G>. This centralizer is computed as a subgroup of the corresponding
normalizer.


%%%%%%%%%%%%%%%%%%%%%%%%%%%%%%%%%%%%%%%%%%%%%%%%%%%%%%%%%%%%%%%
\Section{Q-classes and Z-classes of integer matrix groups}

A $\Q$-class of finite unimodular groups is a maximal set of subgroups
of <GL(n,$\Z$)> which are equivalent under conjugation with elements of
<GL(n,$\Q$)>. Similarly, a $\Z$-class of finite unimodular groups is such 
an equivalence class under conjugation with elements of <GL(n,$\Z$)>.
Since every finite rational matrix group is conjugate in <GL(n,$\Q$)> 
to an integer matrix group, a $\Q$-class can also be defined by a rational
matrix group representative. Every $\Q$-class decays into finitely many
$\Z$-classes. 

\> ZClassRepsQClass( <G> )

returns a list of representatives of the $\Z$-classes contained in the
$\Q$-class defined by the finite rational matrix group representative <G>.

\> `RepresentativeOperation( GL(<n>,Integers), <G1>, <G2> )'%
{RepresentativeOperation in GL}@{`RepresentativeOperation', in GL(n,Z)}

determines whether the finite unimodular groups <G1> and <G2> are 
conjugate in <GL(n,$\Z$)>. If the groups are conjugate, a conjugating 
matrix <m> in <GL(n,$\Z$)> is returned, such that $m G1 m^{-1} = G2$. 
If the groups are not conjugate, the result is `fail'.


